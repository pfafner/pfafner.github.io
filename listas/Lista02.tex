\documentclass{article}
\usepackage{amsfonts}
\usepackage{amsmath}
\usepackage{amssymb}
\usepackage{pictex}
\usepackage[dvips]{graphics}
\usepackage{color}
\usepackage{psfrag}
\usepackage{verbatim}
\usepackage{enumerate}
\usepackage{graphicx}
\usepackage{float}
\usepackage{multicol}

\newcommand {\bc}{\begin{center}}
\newcommand {\ec}{\end{center}}

\newcommand{\CC}{\mathbb{C}}
\newcommand{\DD}{\mathbb{D}}
\newcommand{\HH}{\mathbb{H}}
\newcommand{\NN}{\mathbb{N}}
\newcommand{\PP}{\mathbb{P}}
\newcommand{\QQ}{\mathbb{Q}}
\newcommand{\RR}{\mathbb{R}}
\newcommand{\TT}{\mathbb{T}}
\newcommand{\ZZ}{\mathbb{Z}}

\newcommand{\aaa}{\mathbf{a}}
\newcommand{\bb}{\mathbf{b}}
\newcommand{\cc}{\mathbf{c}}
\newcommand{\ee}{\mathbf{e}}
\newcommand{\nn}{\mathbf{n}}
\newcommand{\pp}{\mathbf{p}}
\newcommand{\qq}{\mathbf{q}}
\newcommand{\ttt}{\mathbf{t}}
\newcommand{\uu}{\mathbf{u}}
\newcommand{\vv}{\mathbf{v}}
\newcommand{\ww}{\mathbf{w}}
\newcommand{\xx}{\mathbf{x}}
\newcommand{\yy}{\mathbf{y}}
\newcommand{\zz}{\mathbf{z}}

% derivadas
\newcommand{\ddx}[1]{\frac{d}{d #1}}
\newcommand{\dfdx}[2]{\frac{d #1}{d #2}}
\newcommand{\ddyxx}[2]{\frac{d^2 #1}{d #2^2}}
\newcommand{\ddydx}[3]{\frac{d^2 #1}{d #2 d #3}}
\newcommand{\ttx}[1]{\tfrac{d}{d #1}}
\newcommand{\tftx}[2]{\tfrac{d #1}{d #2}}

\newcommand{\ppx}[1]{\frac{\partial}{\partial #1}}
\newcommand{\pfpx}[2]{\frac{\partial #1}{\partial #2}}
\newcommand{\ppfxx}[2]{\frac{\partial^2 #1}{\partial #2^2}}
\newcommand{\ppfpx}[3]{\frac{\partial^2 #1}{\partial #2 \partial #3}}
\newcommand{\ssx}[1]{\tfrac{\partial}{\partial #1}}
\newcommand{\sfsx}[2]{\tfrac{\partial #1}{\partial #2}}
\newcommand{\ssfxx}[2]{\tfrac{\partial^2 #1}{\partial #2^2}}
\newcommand{\ssfpx}[3]{\tfrac{\partial^2 #1}{\partial #2 \partial #3}}

\DeclareMathOperator{\argmin}{argmin}
\DeclareMathOperator{\argmax}{argmax}
\DeclareMathOperator{\Aut}{Aut}
\DeclareMathOperator{\diam}{diam}
\DeclareMathOperator{\dist}{dist}
\DeclareMathOperator{\divv}{div}
\DeclareMathOperator{\im}{Im}
\DeclareMathOperator{\idx}{Index}
\DeclareMathOperator{\Int}{Int}
\DeclareMathOperator{\Ker}{Ker}
\DeclareMathOperator{\proj}{proj}
\DeclareMathOperator{\rank}{rank}
\DeclareMathOperator{\re}{Re}
\DeclareMathOperator{\tr}{tr}


\pagestyle{empty}
\setlength{\oddsidemargin}{0pt}
\setlength{\textwidth}{543pt}
\setlength{\marginparsep}{0pt}
\setlength{\marginparwidth}{60pt}
\setlength{\topmargin}{0pt}
\setlength{\headheight}{0pt}
\setlength{\headsep}{0pt}
\setlength{\textheight}{650pt}
\setlength{\footskip}{0pt}
\setlength{\hoffset}{-40pt}

\begin{document}
\begin{sf}
\bc{\Large Teoría de Números 2021} \ec
\bc{Lista 02} \ec
\bc{23.julio.2021} \ec

\bigskip

\begin{enumerate}


% Ejercicio 1
\item
(Entregar sólo (a) y (c)). Para cualesquiera $a, b, c \in \NN$, valen
\begin{multicols}{2}
\begin{itemize}
	\item[a)] $([a, b], [a, c]) = [a, (b,c)]$.
	\item[b)] $[(a, b), (a, c)] = (a, [b,c])$.
	\item[c)] $(ab,ca,bc)(a,b,c) = (a,b)(c,a)(b,c)$.
	\item[d)] $[ab,ca,bc][a,b,c] = [a,b][c,a][b,c]$.
\end{itemize}
\end{multicols}

\bigskip


% Ejercicio 2
\item
	Sea $F_n$ la secuencia de números de Fibonacci, $F_1=1$, $F_2=2$, $F_n = F_{n-1} + F_{n-2}$, para $n \geq 3$. \\
	Muestre que para todo $n \geq 1$, si $a = F_n$ y $b = F_{n+1}$, entonces el algoritmo de Euclides para encontrar $(a,b)$ ejecuta exactamente $n$ divisiones.

\bigskip


% Ejercicio 3
\item
¿Cuáles de las siguientes ecuaciones tienen solución entera? En caso afirmativo, encuentre una solución de dicha ecuación.
\begin{itemize}
	\item[i)] $6x + 51y = 22$.
	\item[ii)] $14x + 35y = 93$.
	\item[iii)] $33x + 14y = 115$.
\end{itemize}
\bigskip


% Ejercicio 4
\item
Determine todos los pares ordenados $(a,b) \in \NN \times \NN$ tales que el menor múltiplo común de $a$ y $b$ es $2^3 \cdot 5^7 \cdot 11^{13}$.

\bigskip


% Ejercicio 5
\item
Asumiendo que $(a, b) = 1$, pruebe los siguientes:
\begin{itemize}
	\item[a)] $(a + b, a - b) = 1$ ó $2$.
	\item[b)] $(2a + b, a+ 2b) = 1$ ó $3$.
	\item[c)] $(a + b, a^2 + b^2) = 1$ ó $2$.
\end{itemize}

\bigskip


% Ejercicio 6
\item
Para $n \geq 1$, y $a,b \in \ZZ^+$, muestre que
\begin{itemize}
	\item[a)] Si $(a, b) = 1$, entonces $(a^n, b^n) = 1$.
	\item[b)] Si $a^n \mid b^n$, entonces $a \mid b$.
\end{itemize}

\bigskip


% Ejercicio 7
\item
¿Cuál es la probabilidad de que al elegir al azar un divisor positivo de $2021^{99}$, éste sea un múltiplo de $2021^{77}$?

\bigskip


% Ejercicio 8
\item
Un entero se llama \textbf{libre de cuadrados} si no es divisible por el cuadrado de ningún entero. 
\begin{itemize}
	\item[a)] Pruebe que un entero $n > 1$ es un cuadrado si, y sólo si, en la factoración en primos canónica de $n$ todos los exponentes son pares.
	\item[b)] Muestre que $n > 1$ es libre de cuadrados si, y sólo si, admite una factoración como producto de primos distintos.
	\item[c)] Todo entero $n > 1$ es el producto de un cuadrado perfecto, y un entero libre de cuadrados.
	\item[d)] Verifique que todo entero $n \in \ZZ$ puede expresarse en la forma $n = 2^k m$, donde $k \geq 0$ y $m$ es un número impar.
\end{itemize}

\bigskip


% Ejercicio 9
\item
Determine si el número $701$ es primo.

\bigskip


% Ejercicio 10
\item
Si $n > 1$ no es primo, entonces $M_n = 2^n - 1$ no es un primo de Mersenne. Esto es, muestre que si $d \mid n$, entonces $2^d - 1 \mid 2^n -1$. \\
Verifique que $2^{35} - 1$ es divisible por $31$ y por $127$.


\end{enumerate}

\underline{\hspace{15cm}}

\bigskip
\noindent 

\end{sf}
\end{document}
