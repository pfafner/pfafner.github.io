\documentclass{article}
\usepackage{amsfonts}
\usepackage{amsmath}
\usepackage{amssymb}
\usepackage{pictex}
\usepackage[dvips]{graphics}
\usepackage{color}
\usepackage{psfrag}
\usepackage{verbatim}
\usepackage{enumerate}
\usepackage{graphicx}
\usepackage{float}
\usepackage{hyperref}

\newcommand {\bc}{\begin{center}}
\newcommand {\ec}{\end{center}}

\newcommand {\EE}{\mathbb{E}}
\newcommand {\JJ}{\mathbb{J}}
\newcommand {\NN}{\mathbb{N}}
\newcommand {\PP}{\mathbb{P}}
\newcommand {\RR}{\mathbb{R}}
\newcommand {\VV}{\mathbb{V}ar}
\newcommand {\XX}{\mathbb{X}}
\newcommand {\ZZ}{\mathbb{Z}}

\pagestyle{empty}
\setlength{\oddsidemargin}{0pt}
\setlength{\textwidth}{543pt}
\setlength{\marginparsep}{0pt}
\setlength{\marginparwidth}{60pt}
\setlength{\topmargin}{0pt}
\setlength{\headheight}{0pt}
\setlength{\headsep}{0pt}
\setlength{\textheight}{650pt}
\setlength{\footskip}{0pt}
\setlength{\hoffset}{-40pt}

\begin{document}
\begin{sf}
\bc{\Large Ciencia de Datos 2021} \ec
\bc{Lista 02} \ec
\bc{13.febrero.2021} \ec

\bigskip

\begin{enumerate}

% Ejercicio 1
\item
Para un estudio se mide la temperatura en diferentes posiciones del
cuerpo de una muestra de personas. Un investigador expresa todas las
temperaturas en grados Celcius. Otro investigador convierte primero
todas estas temperaturas a grados Fahrenheit. \\

¿Cómo se relacionan las matrices de covarianza de sus datos? \\
Si ambos deciden proyectar en la dirección de máxima varianza, ¿obtendrán las mismas direcciones de proyección? Explica tu respuesta.

\bigskip


% Ejercicio 2
\item 
Dibuja un ejemplo de una distribución (2D) donde cualquier proyección
es de varianza máxima.

\bigskip


% Ejercicio 3
\item 
Sea $X = (X_1,X_2,\ldots,X_d)$ una v.a. multidimensional con matriz de covarianza $Cov(X)$ y
$\EE(X) = 0$. \\
Si $\mathbf{l} = (l_1, l_2, \ldots, l_d)^T \in \RR^d$ es un autovector de $Cov(X)$ con autovalor $\lambda$ y $Y = \langle \mathbf{l}, X \rangle$ muestra que $Cov(Y, X_i) = \lambda l_i$.

\bigskip


% Ejercicio 4
\item 
Sea $\XX$ una matriz arbitraria $n \times d$ y $\JJ$ la matriz $n \times n$ para centrar $\XX$. Verifica que
\begin{itemize}
	\item[a)] $\JJ$ es una matriz de proyección.
	\item[b)] el vector $\mathbf{1} = (1,\ldots,1)^T$ de longitud $n$ es un autovector de $\JJ$ con autovalor 0.
\end{itemize}

\bigskip


% Ejercicio 5
\item 
Considera los datos \texttt{weather.csv}. Se trata de los promedios mensuales de
la temperatura (en Celsius) en 35 estaciones canadienses de monitoreo.
El interés es comparar las estaciones entre sí con base en sus curvas de
temperatura. \\
Considerando las 12 mediciones por estación como un vector $X$, aplica
un análisis de componentes principales. Como $X$ representa (un muestreo
de) una curva, este tipo de datos se llama datos funcionales. Interpreta
y dibuja (como curva) los primeros dos componentes, $p_1$, $p_2$ es decir
grafica $\{(i, p_{1i})\}$ y $\{(i, p_{2i})\}$. Agrupa e interpreta las estaciones en el biplot (ten en mente un mapa de Canadá).

\bigskip

	
% Ejercicio 6
\item
A partir de una base de datos con actos delictivos en EE.UU (1970), se construyó la tabla con las correlaciones entre la ocurrencia de 7 clases de delitos, como aparece en la tabla \texttt{crimes.dat}. \\
Consideramos cada clase de delito como una observación. Podemos medir la distancia entre dos observaciones como $1$ menos su correlación (las correlaciones en la tabla son siempre positivas). Así, la distancia mínima (0) corresponde a correlación máxima (1) entre las variables correspondientes.  \\

Encuentra una visualización usando escalamiento multidimensional para estas observaciones y busca una interpretación del eje principal.

\bigskip



% Ejercicio 7
\item
Históricamente uno de los primeros usos de PCA en el área de procesamiento de imágenes fue como método de compresión. Para ello, se divide la imagen en bloques de $c \times c$ pixeles (por ejemplo, tome $c$ un denominador común de las dimensiones de la imagen). Con los valores de los pixeles en cada bloque se forma un vector $(x_1, x_2, \ldots, x_{c^2}) \in \RR^{c^2}$. La matriz de datos se forma con todos estos vectores provenientes de los bloques vectorizados. La compresión consiste en proyectar los datos sobre los primeros $k$ componentes principales. La decompresión consiste en reconstruir los datos a partir de estas proyecciones. \\

Implementa lo anterior para algunas imágenes sencillas (en escala de gris) y muestra el efecto del valor de $k$ sobre la calidad de la reconstrucción. %No olvidar que los datos no son centrados.
\end{enumerate}


\underline{\hspace{15cm}}

\bigskip
\noindent 

\end{sf}
\end{document}
